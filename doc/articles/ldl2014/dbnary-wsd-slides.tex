\documentclass{beamer}

\usepackage{beamerthemesplit}

\usetheme{Boadilla} 

\usepackage[utf8]{inputenc}
\usepackage[T1]{fontenc}
\usepackage[utf8]{vietnam}
\usepackage[english]{babel}
\usepackage{calc}
\usepackage{ifthen}
\usepackage{tikz}
\usepackage{eurosym}

\usepackage{alltt}

\usepackage{verbatim}

\usepackage[french,algoruled,vlined,noend]{algorithm2e}
\SetKwIF{Si}{SinonSi}{Sinon}{si}{}{sinon si}{sinon}{}
\SetKwFor{Pour}{pour}{}{}
\SetKwFor{RepeterNFois}{répéter}{fois}{}
\SetKwFor{TantQue}{tant que}{}{}%
\SetKwFor{PourChaque}{pour chaque}{}{}%

\usepackage{CJKutf8}

\usepackage[overlap, CJK]{ruby}
\usepackage{CJKulem}

\renewcommand{\rubysep}{-0.2ex}

\newenvironment{SChinese}{%
  \CJKfamily{gbsn}%
  \CJKtilde
  \CJKnospace}{}
\newenvironment{TChinese}{%
  \CJKfamily{bsmi}%
  \CJKtilde
  \CJKnospace}{}
\newenvironment{Japanese}{%
  \CJKfamily{min}%
  \CJKtilde
  \CJKnospace}{}
\newenvironment{Korean}{%
  \CJKfamily{mj}}{}
  
\newcommand{\cntext}[1]{\begin{CJK}{UTF8}{}\begin{Japanese}#1\end{Japanese}\end{CJK}}

\pgfdeclareimage[height=0.8cm]{logo}{img/getalp}
\pgfdeclareimage[height=1.5cm]{logobig}{img/getalp} 

\logo{\pgfuseimage{logo}}

\title[Disambiguating Translations in DBnary]{Attaching Translations to Proper Lexical Senses in DBnary}
%\subtitle{Wiktionnaire, RDF, Linked Data: une colonne vertébrale pour le lexique ?}
\author[Tchechmedjiev et al.]{Andon Tchechmedjiev, Gilles Sérasset, Jérôme Goulian, Didier Schwab}
\institute[GETALP-LIG, Grenoble]{GETALP-LIG, Univ Grenoble Alpes, France}
\titlegraphic{\pgfuseimage{logobig}}
\date{May 27, 2014}

% Delete this, if you do not want the table of contents to pop up at
% the beginning of each subsection:
\AtBeginSection[]
{
  \begin{frame}<beamer>
    \frametitle{}
    \tableofcontents[currentsection]
  \end{frame}
}

\begin{document}

\frame{\titlepage}


\section{Introduction}



\section{Overview of the DBnary dataset}

\frame{\frametitle{What is Dbnary?}
\begin{itemize}
\item Extracting lexical data from 12 different Wiktionary language editions
	\begin{itemize} 
	\item Wiktionary
	\end{itemize}
\item Intéropérabilité "à grain fin"\\
	\begin{itemize} 
	\item RDF
	\end{itemize}
\item Un modèle standard
	\begin{itemize} 
	\item LEMON (inspiré de LMF)
	\end{itemize}
\item Multilingue
	\begin{itemize} 
	\item extraction de plusieurs "language edition" de Wiktionary
	\end{itemize}
\item Disponible, utilisable et utilisé
	\begin{itemize} 
	\item Linked Data
	\end{itemize}
	
\end{itemize}
}

\frame{  
\frametitle{Dbnary: Wiktionary comme graphe lexical}

     \includegraphics<1>[width=\linewidth]{img/cat1.png}
     \includegraphics<2>[width=\linewidth]{img/cat2.png}
     \includegraphics<3>[width=\linewidth]{img/cat3.png}
     \includegraphics<4>[width=\linewidth]{img/cat4.png}
     \includegraphics<5>[width=\linewidth]{img/cat5.png}
     \includegraphics<6>[width=\linewidth]{img/cat6.png}
     \includegraphics<7>[width=\linewidth]{img/cat_struct}
     \includegraphics<8>[width=\linewidth]{img/cat_syn}
     \includegraphics<9>[width=\linewidth]{img/cat_translations}

}

\frame{  
\frametitle{LEMON}

     \includegraphics<1>[width=\linewidth]{img/lemon-core.png}
     \includegraphics<2>[width=\linewidth]{img/lemon-descr.png}
     \includegraphics<3>[width=\linewidth]{img/dbnary-lemon-xtension.pdf}

}

\frame{
\frametitle{Principe d'extraction}
\begin{itemize}
\item 8 éditions extraites: en, fr, de, ru, it, pt, fi, el
\item Extraction à partir des dumps de wikimedia
\item Chaque édition met à jour son dump tous les 10-15j
\item l'extraction est faite en direct, les fichiers sont accessibles en ligne immédiatement
\item le serveur "Linked Data" contient des données plus anciennes
\end{itemize}
}

\frame{  
\frametitle{Taille des données (aujourd'hui)}

\begin{table}[htb]
\begin{tabular}{lrrrr}
 & \textbf{Entries} & \textbf{Vocables} & \textbf{Senses} & \textbf{Translations}\\
 \hline
\textbf{eng} & 527067 & 504594 & 421232 & 1126463 \\
\textbf{fra} & 273822 & 283847 & 358921 & 464956 \\
\textbf{deu} & 135103 & 201736 & 95593 & 471892 \\
\textbf{rus} & 127271 & 139235 & 99243 & 325345 \\
\textbf{ell} & 74056 & 74800 & 34932 & 55652 \\
\textbf{fin} & 48164 & 48050 & 56559 & 118728 \\
\textbf{por} & 43042 & 44061 & 77631 & 225065 \\
\textbf{ita} & 25279 & 31935 & 35061 & 57796 \\
\end{tabular}
\caption{Number of resources by type and language, sorted by number of lexical entries.}\label{globalsize}
\end{table}
}

\frame{  
\frametitle{Taille des données (aujourd'hui)}
\begin{table}[htb]
\begin{tabular}{lrrrrrr}
 & \textbf{syn}  & \textbf{ant} & \textbf{hyper} & \textbf{hypo} & \textbf{mero} & \textbf{holo} \\
 \hline
\textbf{eng} & 31461& 6877& 959& 1103& 114& 0 \\ 
\textbf{fra} & 30088& 6735& 8215& 3557& 943& 1847 \\ 
\textbf{deu} & 27516& 14315& 30202& 9509& 0& 0 \\ 
\textbf{rus} & 22631& 9204& 21028& 4756& 0& 0 \\ 
\textbf{ell} & 3975& 1116& 0& 0& 0& 0 \\ 
\textbf{fin} & 2255& 0& 0& 0& 0& 0 \\ 
\textbf{por} & 3527& 575& 6& 3& 0& 0 \\ 
\textbf{ita} & 7091& 2337& 0& 0& 0& 0 \\ 
\end{tabular}
\caption{Number of lexicon-semantic relations. Languages are sorted according to their number of lexical entries.}\label{nymsize}
\end{table}
}

\frame{  
\frametitle{Taille des données (aujourd'hui)}
\begin{table}[htb]
\begin{tabular}{lrrrrrrrrrrrrr}
\textbf{Source/Target}  & \textbf{deu} & \textbf{ell} & \textbf{eng} & \textbf{fin} & \textbf{fra} & \textbf{ita} & \textbf{por} & \textbf{rus}& \textbf{others} & \textbf{Total} & \textbf{\# of target languages}\\
\textbf{eng} & 62501 & 23794 & 1 & 74938 & 57959 & 37467 & 30256 & 74837 & 764710 & 1126463 & 1143\\
\textbf{fra} & 34608 & 7063 & 74687 & 7589 & 12 & 18806 & 17784 & 7783 & 296624 & 464956 & 952\\
\textbf{deu} & 0 & 2675 & 81015 & 4947 & 67143 & 41485 & 8872 & 17354 & 248401 & 471892 & 355\\
\textbf{rus} & 23056 & 3295 & 48559 & 3966 & 14776 & 12643 & 5567 & 0 & 206709 & 318571 & 490\\
\textbf{ell} & 2242 & 2 & 10090 & 1056 & 8436 & 1470 & 1149 & 1315 & 29892 & 55652 & 246\\
\textbf{fin} & 8046 & 918 & 30103 & 0 & 6700 & 3856 & 2196 & 7997 & 58912 & 118728 & 329\\
\textbf{por} & 7000 & 2816 & 11284 & 4607 & 8720 & 7096 & 4 & 4396 & 179142 & 225065 & 695\\
\textbf{ita} & 4619 & 506 & 17539 & 925 & 4461 & 75 & 1219 & 938 & 27514 & 57796 & 315\\
\end{tabular}
\caption{Number of translations from/to the 8 currently extracted languages. Source languages are sorted according to their number of lexical entries. Target languages are sorted by their ISO 639-3 language code. The number of different target languages is also given.}\label{tradsize1}
\end{table}
}

\frame{  
\frametitle{Taille des données (aujourd'hui)}
\begin{table}[htb]
\begin{tabular}{lrrrrrr}
\textbf{Source/Target}  & \textbf{por} & \textbf{rus}& \textbf{others} & \textbf{Total} & \textbf{\# of lang}\\
\textbf{eng} & 30256 & 74837 & 764710 & 1126463 & 1143\\
\textbf{fra} & 17784 & 7783 & 296624 & 464956 & 952\\
\textbf{deu} & 8872 & 17354 & 248401 & 471892 & 355\\
\textbf{rus} &  5567 & 0 & 206709 & 318571 & 490\\
\textbf{ell} &  1149 & 1315 & 29892 & 55652 & 246\\
\textbf{fin} &  2196 & 7997 & 58912 & 118728 & 329\\
\textbf{por} &  4 & 4396 & 179142 & 225065 & 695\\
\textbf{ita} &  1219 & 938 & 27514 & 57796 & 315\\
\end{tabular}
\caption{Number of translations from/to the 8 currently extracted languages. Source languages are sorted according to their number of lexical entries. Target languages are sorted by their ISO 639-3 language code. The number of different target languages is also given.}\label{tradsize2}
\end{table}
}

\frame{  
\frametitle{Qualité des données}

\begin{itemize}
\item Évaluer la qualité des données est difficile
\item Qualité de l'extraction $\ne$ qualité des données
\item Pas d'application particulière actuellement
\item Pas encore d'alignement avec d'autres ressources (wordnet, jeuxdemots, ...)
\item Mais quelques indices...
\end{itemize}
}

\frame{  
\frametitle{Qualité des données (hier)}
\begin{table}[htb]
\begin{tabular}{lrr}
\textbf{language} & \textbf{\# of transl.}\\
 \hline
\textbf{eng} & 5110 (99.1 \%) \\
\textbf{fra} & 5799 (107.0 \%) \\
\textbf{deu} & 10287 (99.2 \%)\\
\textbf{rus} & 8436 (24811.7 \%) \\
\textbf{ell} & 2598 (64.3 \%) \\
\textbf{fin} & 7245 (28980 \%) \\
\textbf{por} & 17720 (93.2 \%) \\
\textbf{ita} & 7855 (3167.3 \%) 
\end{tabular}
\caption{Extracted translations vs interwiki links, on a random sample of 1000 entries.}\label{iwlinks}
\end{table}
}

\section{Attaching Translation to their Proper Lexical Sense}

\subsection{Problem}

\frame{  
\frametitle{The problem}

\centering{Use part of Andon's graph of the 1rst year presentation
}

}


\subsection{Selecting a Similarity Measure}

\subsection{Selecting a Similarity Measure}

\section{Evaluation}

\subsection{Extracting an Endogeneous Gold Standard}

\subsection{Results}

\section{Conclusion}

\frame{  
\frametitle{Conclusion}

\begin{itemize}

\end{itemize}
}


\end{document}




